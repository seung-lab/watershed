\documentclass{llncs}

%\usepackage{graphicx}
%
\usepackage[named]{algo}

\begin{document}


\title{Image Segmentation by Size-Dependent Single Linkage
  Clustering of a Watershed Basin Graph}

\subtitle{Supplementary Material}

\author{{*}{*}{*}{*}{*}{*}{*}{*}{*}{*}{*}{*}{*}{*}{*}{*}{*}{*}{*}}


\institute{{*}{*}{*}{*}{*}{*}{*}{*}{*}{*}{*}\\
\texttt{{*}{*}{*}{*}{*}{*}{*}{*}{*}}\\
{*}{*}{*}{*}{*}{*}{*}{*}{*}{*}{*}\\
\texttt{{*}{*}{*}{*}{*}{*}{*}{*}{*}}}

\maketitle


\section{The Watershed Transform Algorithm}\label{sec:Introduction}

We present the watershed transform algorithm from {\bf Section 2} of the main manuscript. We assume there is an ordering of the vertices in $V$. Usually the information about the vertices is stored in an array, where the $i$-th element of the array contains the vertex with index $i$.

\begin{enumerate}
\item Apply the threshold $T_{\min}$ and $T_{\max}$:
  \begin{enumerate}
  \item Remove each $\{u,v\}$ from $E$ if $w(\{v_i,u\}) > T_{\max}$.
  \item For each $\{u,v\}$ from $E$ set $w(\{v_i,u\}) = 0$ if $w(\{v_i,u\}) < T_{\min}$.
  \item Remove singleton vertices (vertices with no incident edges in $E$). Mark them as background.
  \end{enumerate}
\item Create $G'$. Set $V'=V$ and $E'=\emptyset$. For each vertex $v_i \in V'$:
  \begin{enumerate}
  \item Calculate $M_i = \min_{u \in V', \{v_i,u\} \in E}\{w(\{v_i,u\})\}$.
  \item For each $u \in V'$ such that $\{v_i,u\} \in E$ add $(v_i,u)$ to $E'$ if $w(\{v_i,u\})=M_i$.
  \end{enumerate}
\item Modify $G'$ to remove all saddle vertices. For each vertex $v \in V$ that has more than one outgoing edge, keep only one outgoing edge pointing to a vertex with the minimal index.
\item Modify $G'$ to split the non-minimal plateaus:
  \begin{enumerate}
  \item Initialize a FIFO queue $Q$.
  \item For each vertex $v \in V'$, check whether the vertex has at least one outgoing edge and one bidirectional edge (check whether the vertex is a plateau corner). If it does mark it as visited and add it to the end of $Q$.
  \item While $Q \neq \emptyset$, let $u$ be the first element of $Q$ and remove it from $Q$. For all $v$ such that $(u,v) \in E'$ and $(v,u) \in E'$ remove $(u,v)$ from $E'$. If $v$ is visited remove $(v,u)$ from $E'$ as well, otherwise mark $v$ as visited and add it to the end of $Q$.
  \end{enumerate}
\item Replace all unidirectional edges with bidirectional edges. For each $(u,v) \in E'$ add $(v,u)$ to $E'$ if not already there.
\item Return connected components of the modified $G'$.
\end{enumerate}

In the first step of the algorithm we apply the thresholds and mark the singleton vertices as background, as desired by the output. We expect all the other vertices to be assigned to a watershed basin. In the second step we create the steepest descent graph. In the steps 3 we make sure that all the saddle vertices are removed. After this step there will be no vertices with more than one outgoing edge. The step 4 splits the plateaus. In the breadth first search, while examining the plateau vertices, we make sure that all the vertices have only one outgoing edge. The breadth first search order will ensure that we keep the edge on the path to the closest plateau corner (with the minimal index). After the step 4 all vertices that don't belong to a regional minima will have a single outgoing edge, hence there will be an unique steepest descent path to a single regional minima. Therefore, all the vertices will be uniquely assigned to a basin. The step 5 just makes sure there's also a path from a regional minima to all vertices in its basin of attraction so that we can just apply connected components in order to obtain the watershed basins.

As we operate on a connected graph we assume $O(|E|) \ge O(|V|)$. The first three steps of the algorithm visit each edge constant number of times. In the step 4, we visit each vertex at most once. While visiting each vertex we visit edges incident to it constant number of times. The step 5 also visits each edge at most once. As the complexity of connected components is $O(|E|)$, we get our overall complexity to be $O(|E|)$.

\subsubsection{Equivalence of $T_{\min}$.} In the manuscript we claim that a segmentation obtained by:
\begin{enumerate}
\item Applying watershed transform and then merging all basins with saliency smaller than $T_{\min}$, and
\item Applying watershed on a graph where with all the edges with weight smaller than $T_{\min}$ are replaced with edges with weight of $0$
\end{enumerate}
are equivalent. We first prove the following lemma:

\subsubsection{Lemma 1} \emph{Examine a watershed transform $W$ of $G$ and watershed transform $W'$ obtained by applying a threshold $T_{\min}$ on $G$. Let $\{B_1,B_2,\dots\}$ be the watershed basins of $W$ and $\{B'_1,B'_2,\dots\}$ the watershed basins of $W'$. For each $B_i$ there exist $B'_j$ such that $B_i \subseteq B'_j$.}

\subsubsection{Proof.} The steepest descent graph containing only vertices in $V$ that are not incident to any edge with the weight smaller than $T_{\min}$ will stay unmodified, even after splitting the plateaus and getting rid of the saddle vertices. All other vertices will became a part of a regional minima (locally minimal plateau in the steepest descent graph). All previous locally minimal edges will stay locally minimal. Therefore applying the threshold $T_{\min}$ just introduces bidirectional edges to the steepest descent graph. Therefore, a connected component in the modified steepest descent graph of $G$ has to be a subset of a connected component of the modified steepest descent graph of $G$ after applying $T_{\min}$.

Watershed basins of $W'$ are connected. Hence, they can be obtained by merging watershed basins of $W$. We prove that the result of 1. and 2. are equivalent by showing that two neighboring basins of $W$, $B_i$ and $B_j$ will be merged in 1. if and only if they are merged in 2.

Let's examine two watershed basins of $W$, $B_i$ and $B_j$. If there is an edge $\{u,v\}$ in $G$ such that $u \in B_i$ and $v \in B_j$ with the weight smaller than $T_{\min}$, then $u$ and $v$ will be part of the same regional minima in $W'$. Hence, the two basins of $W$ belong to a single basin of $W'$. We also know that the saliency between $B_i$ and $B_j$ has to be smaller than $T_{\min}$, therefore they will belong to the same segment when applying the algorithm 1. Similarly, if the saliency $d_{ij}$ between $B_i$ and $B_j$ is below $T_{\min}$, then there exist an edge $\{u,v\}$ in $G$ such that $u \in B_i$ and $v \in B_j$ with the weight equal to $d_{ij} < T_{\min}$. Therefore $u$ and $v$ will be part of the same regional minima, and $B_i$ and $B_j$ will be part of the same watershed basin in $W'$.\qed


\section{Hierarchical Clustering of the Watershed Basin Graph}\label{sec:Others}

The following algorithm and theorem were introduced in the main manuscript.

\subsubsection{Algorithm 1 } Size-dependent single linkage clustering
\begin{enumerate}
\item Order $E_{W}$ into $\pi(e_{1},\dots,e_{n})$, by non-decreasing edge
weight.
\item Start with basins as singleton clusters $S^{0}=\{C_{1}=\{B_{1}\},C_{2}=\{B_{2}\},\dots\}$
\item Repeat step $4$ for $k=1,\dots,n$
\item Construct $S^{k}$ from $S^{k-1}$. Let $e_{k}=\{B_{i},B_{j}\}$ be
the $k$-th edge in the ordering. Let $C_{i}^{k-1}$ and $C_{j}^{k-1}$
be components of $S^{k-1}$ containing $B_{i}$ and $B_{j}$. If $C_{i}^{k-1}\neq C_{j}^{k-1}$
and $\Lambda(C_{i}^{k-1},C_{j}^{k-1})$ is not satisfied then $S^{k}$ is created from
$S^{k-1}$ by merging $C_{i}^{k-1}$ and $C_{j}^{k-1}$, otherwise
$S^{k}=S^{k-1}.$
\item Return the hierarchical segmentation $(S^{0},\dots,S^{n})$
\end{enumerate}
\textbf{Theorem 1 }\emph{The highest level of the hierarchical segmentation
produced by algorithm (1) will have the predicate $\Lambda$ satisfied
for all pairs of the clusters. The complexity of the algorithm is
$\left|E_{W}\right|\log\left|E_{W}\right|$. The algorithm can be
modified to consider only the edges of the minimum cost spanning tree
of $G_{W}$.}

\subsubsection{Proof.} First we prove that $\Lambda$ will be satisfied for all pairs of the clusters in the highest level. Assume $\Lambda(C^n_i,C^n_j)$ is not satisfied for some pair of clusters, $d_{C^n_i,C^n_j} < \tau(\min\{S(C^n_i),S(C^n_j)\})$. That means there exist an edge $e \in E_W$ with the weight equal to $d_{C^n_i,C^n_j}$. Without losing generality assume $e$ was examined at the $(k+1)$-th step of the algorithm and the following predicate was satisfied: $d_{C^k_p,C^k_q} \ge \tau(\min\{S(C^k_p),S(C^k_q)\})$. As $C^k_p \subseteq C^n_i$ and $C^k_q \subseteq C^n_j$ then $\min\{S(C^k_p),S(C^k_q)\} \le \min\{S(C^n_i),S(C^n_j)\}$ and $\tau(\min\{S(C^k_p),S(C^k_q)\}) \ge \tau(\min\{S(C^n_i),S(C^n_j)\}) > d_{C^n_i,C^n_j} = d_{C^k_p,C^k_q}$. Contradiction. \qed

Assuming real values of the weights, the complexity of the algorithm is dominated by sorting the edges of $E_W$. If we use \emph{union-find} data structure, an iteration of the step 4. takes $O(u(n))$, where $n$ is the number of basins and $u$ is the inverse Ackermann function.

Let $E_T$ be the set of edges on an MST of the watershed basin graph. Assume we run the previous algorithm considering only edges in $E_T$. We want to prove that $\Lambda$ will be satisfied for all pairs of the clusters at the highest level. Again, assume $\Lambda(C^n_i,C^n_j)$ is not satisfied for some pair of clusters, $d_{C^n_i,C^n_j} < \tau(\min\{S(C^n_i),S(C^n_j)\})$. That means there exist an edge $e \in E_W$ with the weight equal to $d_{C^n_i,C^n_j}$. If $e \in E_T$, then $e$ was examined in the algorithm. We follow the same steps as above to show that when $e$ was examined the predicate must have been unsatisfied meaning that $C^n_i$ and $C^n_j$ can't be two different clusters. If $e = \{B_p,B_q\}$ is not on the MST, then there is a path in the watershed basin graph from $B_p$ to $B_q$ where all the edges on the path have the weight less or equal to the weight of $e$. Without losing generality, assume $S(C^n_i) \le S(C^n_j)$ and $B_p \in C^n_i$. Then $\tau(S(C^n_j)) > w(e)$. Let $e_x=\{B_a,B_b\}$ be an edge on the path from $B_p$ to $B_q$ in the MST such that $B_a \in C^n_i$ and $B_b \notin C^n_i$. Such edge must exist, otherwise $B_p$ and $B_q$ would be part of the same cluster. As $e_x$ is in the MST, it was examined in the algorithm. When $e_x$ was examined the size of the cluster $C$ that $B_a$ was part of was smaller or equal than $S(C^n_j))$. Hence $\tau(S(C)) \ge \tau(S(C^n_j)) > w(e) \ge w(e_x)$. Therefore $e_x$ can't exist, and $B_p$ and $B_q$ have to be part of the same component. Contradiction. \qed



\end{document}
